% Beamer do material do curso de Verão (2015) do IME-USP
% Introdução ao Projeto de Jogos
%
% Baseado no template LaTeX das apresentações do LIDET versão 2
% (https://github.com/luigivieira/LIDET)
%

\providecommand\classopts{}
\expandafter\documentclass\expandafter[table, usenames, svgnames, dvipsnames,%
                                       \classopts]{beamer}
\usepackage{etex}
\usepackage{beamerthemeshadow}
\usepackage[portuguese]{babel}
\usepackage[utf8]{inputenc}
\usepackage[absolute,overlay]{textpos}
\usepackage{array}
\usepackage{framed}
\usepackage{booktabs}
\usepackage[compatibility=false]{caption}
\usepackage{subcaption}
\usepackage{outlines}
\usepackage{ulem}
\usepackage{xcolor,colortbl}
\usepackage{ragged2e}
\usepackage{tikz}

% ---------------------------------------------------------------------------- %
% Presentation definitions
% ---------------------------------------------------------------------------- %
\usetheme{Luebeck}
\hypersetup{pdfpagemode=FullScreen} % Starts the presentation in full screen

% layout
\setbeamerfont{frametitle}{size=\normalsize}
\setbeamerfont{title}{size=\normalsize}
\beamertemplatenavigationsymbolsempty
\setbeamertemplate{bibliography item}[text]%
\setbeamertemplate{headline}{} % Remove a barra superior.

% colors
\definecolor{lidet_orange}{rgb}{0.9, 0.49, 0.09}
\definecolor{lidet_black}{rgb}{0.2, 0.2, 0.2}

\setbeamercolor{title}{bg=lidet_orange}
\setbeamercolor{structure}{bg=white, fg=lidet_orange}
\setbeamercolor{normal text}{fg=black}
\setbeamercolor{section in head/foot}{fg=white, bg=lidet_black}
\setbeamercolor{postit}{fg=white, bg=lidet_orange!90!lidet_black}

% shadow
\makeatletter
\pgfdeclareverticalshading[black,bg]{bmb@shadow}{200cm}{%
    color(0bp)=(lidet_black!25);%
    color(4bp)=(black!50!bg);%
    color(8bp)=(black!50!bg)%
}
\pgfdeclareradialshading[black,bg]{bmb@shadowball}{\pgfpointorigin}{%
    color(0bp)=(black!50!bg); color(4bp)=(lidet_black!25)%
}
\pgfdeclareradialshading[black,bg]{bmb@shadowballlarge}{\pgfpointorigin}{%
    color(0bp)=(black!50!bg);%
    color(4bp)=(black!50!bg);%
    color(8bp)=(lidet_black!25)%
}
\makeatother

% Captions for images and tables
\setlength{\abovecaptionskip}{5pt plus 5pt minus 5pt}
\setlength{\belowcaptionskip}{5pt plus 5pt minus 5pt}
\captionsetup[figure]{font=scriptsize,labelfont=scriptsize}
\captionsetup[table]{font=scriptsize,labelfont=scriptsize}
\captionsetup{labelformat=empty,labelsep=none}

% Dimensions for table rules
\setlength\heavyrulewidth{0.1em} 
\setlength\lightrulewidth{0.01em}
\setlength\belowrulesep{0.10ex}
\setlength\aboverulesep{0.10ex}

% Define macros to mark the begining and ending of references
% Basically, handles the automatically spanned frames (due to parameter
% allowframebreaks)
% as backup frames, so they do not influence in the frame numbering
\newcommand{\referencesbegin}{
   \newcounter{framenumberappendix}
   \setcounter{framenumberappendix}{\value{framenumber}}
}
\newcommand{\referencesend}{
   \addtocounter{framenumberappendix}{-\value{framenumber}}
   \addtocounter{framenumber}{\value{framenumberappendix}} 
}

% Section frames (that appear before each section)
\AtBeginSection[] 
{
	{
        % Hide the footline locally for these frames
		\setbeamertemplate{footline}{}
		\begin{frame}<beamer>[noframenumbering]
			\begin{center}
				\begin{tikzpicture}
					\node[align=left, left color=lidet_orange,%
                          right color=lidet_orange, draw, rounded corners,%
                          minimum width=10cm, minimum height=1cm]%
                          {\color{white} \textbf{\insertsectionhead}};
				\end{tikzpicture}
			\end{center}
			\footnotesize{\tableofcontents[currentsection,%
                          hideothersubsections]}
		\end{frame}
	}
}

\DeclareGraphicsExtensions{.pdf,.jpg,.png}
\graphicspath{{./images/}}

% ---------------------------------------------------------------------------- %
% Presentation title, author and institution
% ---------------------------------------------------------------------------- %
\newcommand{\lessontitle}{Aula 3 - Pensando no gameplay}
\title{\textbf{Introdução ao Projeto de Jogos}}
\subtitle{{\small \lessontitle}}

\newcommand{\autores}{Luiz C. Vieira, Vinícius K. Daros}
\author[\autores]{\scriptsize
    Luiz Carlos Vieira e Vinícius Kiwi Daros\\
    \{luigivieira,vinicius.k.daros\}@gmail.com
}

\newcommand{\lidet}{LIDET (IME - USP)}
\institute[\lidet]{\\[1.0mm] 
Curso de Verão (2015)\\
Departamento de Ciência da Computação}

\date{{\tiny 13 de Janeiro de 2015}}

% ---------------------------------------------------------------------------- %
% Presentation content
% ---------------------------------------------------------------------------- %

% ---------------------------------------------------------------------------- %
\begin{document}
% ---------------------------------------------------------------------------- %

% ---------------------------------------------------------------------------- %
% First Slide (index 0) = cover
% ---------------------------------------------------------------------------- %

{%\usebackgroundtemplate{}} 
\begin{frame}[plain, noframenumbering]
	\begin{columns}[c]
		\column{0.2\textwidth}
			\hspace*{-1.5em}
			\includegraphics[width=0.35\paperwidth]{side_bar}\\
		\column{0.01\textwidth}
		\column{0.70\textwidth}
			\titlepage
			\hspace*{+0.5em}
			\begin{center}
				\includegraphics[height=1.0cm]{lidet-logo}\\
				\includegraphics[height=1.0cm]{ime-logo}\\
			\end{center}
	\end{columns}
	%\addtocounter{framenumber}{-1}
\end{frame}
}

% ---------------------------------------------------------------------------- %
% Other Slides (index from 1 onwards)
% ---------------------------------------------------------------------------- %

% setup navigation symbols and footline
\setbeamertemplate{navigation symbols}{}
\makeatletter
\setbeamertemplate{footline}{%
    \leavevmode%
    \hbox{%
        \begin{beamercolorbox}[wd=0.28\paperwidth,ht=4ex,dp=1ex,left,%
                               leftskip=2ex]{author in head/foot}%
            \usebeamerfont{title in head/foot}
            \insertdate\newline%
            \vskip 0.6ex%
            \autores
        \end{beamercolorbox}%
        \begin{beamercolorbox}[wd=0.53\paperwidth,ht=4ex,dp=1ex,center]%
                              {author in head/foot}%
            \usebeamerfont{author in head/foot}\lessontitle%
        \end{beamercolorbox}%
        \begin{beamercolorbox}[wd=0.19\paperwidth,ht=4ex,dp=1ex,right,%
                               rightskip=2ex]{author in head/foot}%
            \insertframenumber{}/\inserttotalframenumber \newline
            \lidet%
        \end{beamercolorbox}%
    }%
    \vskip 4cm%
}
\makeatother

% ---------------------------------------------------------------------------- %
\begin{frame}[plain]
\frametitle{\textbf{Agenda}}
	\hspace*{+4.0em}
	\footnotesize{ \tableofcontents }
\end{frame}


% ---------------------------------------------------------------------------- %
\section{Gameplay vs. Jogabilidade}
% ---------------------------------------------------------------------------- %

% ------------------------------
\begin{frame}{\textbf{Gameplay vs. Jogabilidade}}
    \centering
    \begin{minipage}{7cm}
        \begin{description}
            \item[Gameplay:] ações dentro do jogo
            \vskip 1cm%
            \item[Jogabilidade:] capacidade de jogar bem\\
                                 (usabilidade dentro do jogo)
        \end{description}
    \end{minipage}
\end{frame}


% ---------------------------------------------------------------------------- %
\section{CCC - Character (Personagem), Câmera, Controle}
% ---------------------------------------------------------------------------- %

\subsection{Character (personagem)}
\subsection{Camera (câmera)}
\subsection{Control (controles)}


% ---------------------------------------------------------------------------- %
\section{Jogabilidade}
% ---------------------------------------------------------------------------- %

\subsection{Protejendo o jogador de erros} %...e ajudando-o a corrigí-los
    % -> Save points
\subsection{Consistência}
\subsection{Sequência lógica de desafios}
    % -> Balanceamento de dificuldades


% ---------------------------------------------------------------------------- %
\section{Gráfico de ritmo}
% ---------------------------------------------------------------------------- %

% ------------------------------
\begin{frame}{\textbf{Gráfico de ritmo}}
    \centering
    \begin{minipage}{7cm}
        \begin{itemize}
            \item Descrição sussinta de cada fase
            \vskip 0.3cm%
            \item Permite acompanhar e controlar...
            \begin{itemize}
                \item aparecimento de inimigos novos
                \item aprendizado de novas habilidades
                \item surgimento de novas mecânicas
                \item dinheiro do personagem
                \item etc
            \end{itemize}
            \vskip 0.3cm%
            \item Muito útil para balanceamento
        \end{itemize}
    \end{minipage}
\end{frame}

% ------------------------------
\begin{frame}{\textbf{Gráfico de ritmo}}
    \scriptsize
    \begin{table}[h]
    \begin{tabular}{l p{6cm}}
    \hline
    \textbf{Número do nível}         & 1.2                             \\ \hline
    \textbf{Título/ambiente}         & Tumba dos escravos              \\ \hline
    \textbf{Hora do dia}             & Noite                           \\ \hline
    \textbf{Elementos de história}   & Kha encontra um mapa e sai da pimâmide
                                       por uma passagem secreta feita pelos
                                       escravos                        \\ \hline
    \textbf{Progressão de gameplay}  & Jogador sabe se movimentar e combate
                                       básico                          \\ \hline
    \textbf{Inimigos}                & Escaravelhos rastejantes, escaravelhos
                                       voadores, múmias de escravos (básico),
                                       múmias de escravos (com ferramentas)
                                       \\ \hline
    \textbf{Mecânicas}               & Alavancas, portas escondidas, vazos
                                       quebráveis, plataformas móveis, escritas
                                       nas paredes                     \\ \hline
    \textbf{Perigos}                 & Armadilha de espinhos na parede,
                                       alçapões, lâminas caindo, abismos
                                       \\ \hline
    \textbf{Tempo para completar}    & 15 minutos                      \\ \hline
    \textbf{Habilidades destravados} & Olhar item de perto, pulo duplo,
                                       escorregar                      \\ \hline
    \textbf{Power-ups encontrados}   & Haste de alavanca, recuperar vida,
                                       vagalume, moedas de ouro, moedas de prata
                                       \\ \hline
    \textbf{Tesouros}                & 25 moedas de prata, 5 moedas de ouro
                                       \\ \hline
    \textbf{Esquema de cores}        & Bege (paredes), marrom (terra/chão),
                                       dourado (estátuas)              \\ \hline
    \textbf{Trilha musical}          & lento\_mistério                 \\ \hline
    \textbf{Material extra}          & -                               \\ \hline
    \end{tabular}
    \end{table}
\end{frame}


% ---------------------------------------------------------------------------- %
\section{Átomos de jogos}
% ---------------------------------------------------------------------------- %

% ------------------------------
\begin{frame}{\textbf{Átomos de jogos}}
    \centering
    \footnotesize
    \begin{minipage}{7cm}
        \begin{itemize}
            \item Objetivo Claro
            \item Recompensas e punições
            \item Jogador como Agente de Mudança
            \item Contexto Compreensível (* rever o que é)
            \item Regras Compreensíveis
            \item Habilidade e Progressão
            \item Feedback sobre resultado (visual ou sonoro)
            \item Interface consistente
            \item Desafios e IA (desafios tipo "soma zero", mesmo sem oponentes
                  humanos)
            \item Alternância entre Desafios e Pausas
            \item Balanceamento entre sorte e estratégia
                  %(sem demonizar os extremos - jogo da vida x xadrez)
            \item Visibilidade e localização/orientação
            \item Uso de padrões
            \item Narrativa e fantasia significativas
        \end{itemize}
    \end{minipage}
\end{frame}


% ---------------------------------------------------------------------------- %
\section{Atividade}
% ---------------------------------------------------------------------------- %

% ------------------------------
\begin{frame}{Atividade}
    \centering
    \begin{minipage}{6cm}
        \begin{itemize}
            \item Analisar o balanceamento e ritmo do gameplay do projeto
            \vskip 1cm%
            \item Atualizar o GDD do projeto conforme o necessário
        \end{itemize}
	\end{minipage}
\end{frame}

\end{document}

